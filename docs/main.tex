\documentclass[12pt,a4paper]{article}
\usepackage[utf8]{inputenc}
\usepackage{dsfont} 
\usepackage[polish]{babel}
\usepackage{amsmath}
\usepackage{graphicx}
\usepackage[top=1in, bottom=1.5in, left=1.25in, right=1.25in]{geometry}

\usepackage{subfig}
\usepackage{multirow}
\usepackage{multicol}
\graphicspath{{Imagens/}}
\usepackage{xcolor,colortbl}
\usepackage{float}

\newcommand \comment[1]{\textbf{\textcolor{red}{#1}}}

%\usepackage{float}
\usepackage{fancyhdr} % Required for custom headers
\usepackage{lastpage} % Required to determine the last page for the footer
\usepackage{extramarks} % Required for headers and footers
\usepackage{indentfirst}
\usepackage{placeins}
\usepackage{scalefnt}
\usepackage{xcolor,listings}
\usepackage{textcomp}
\usepackage{color}
\usepackage{verbatim}
\usepackage{framed}

\definecolor{codegreen}{rgb}{0,0.6,0}
\definecolor{codegray}{rgb}{0.5,0.5,0.5}
\definecolor{codepurple}{HTML}{C42043}
\definecolor{backcolour}{HTML}{F2F2F2}
\definecolor{bookColor}{cmyk}{0,0,0,0.90}  
\color{bookColor}

\lstset{upquote=true}

\lstdefinestyle{mystyle}{
	backgroundcolor=\color{backcolour},   
	commentstyle=\color{codegreen},
	keywordstyle=\color{codepurple},
	numberstyle=\numberstyle,
	stringstyle=\color{codepurple},
	basicstyle=\footnotesize\ttfamily,
	breakatwhitespace=false,
	breaklines=true,
	captionpos=b,
	keepspaces=true,
	numbers=left,
	numbersep=10pt,
	showspaces=false,
	showstringspaces=false,
	showtabs=false,
}
\lstset{style=mystyle}

\newcommand\numberstyle[1]{%
	\footnotesize
	\color{codegray}%
	\ttfamily
	\ifnum#1<10 0\fi#1 |%
}

\definecolor{shadecolor}{HTML}{F2F2F2}

\newenvironment{sqltable}%
{\snugshade\verbatim}%
{\endverbatim\endsnugshade}

% Margins
\addtolength{\footskip}{0cm}
\addtolength{\textwidth}{1.4cm}
\addtolength{\oddsidemargin}{-.7cm}

\addtolength{\textheight}{1.6cm}
%\addtolength{\topmargin}{-2cm}

% paragrafo
\addtolength{\parskip}{.2cm}

% Set up the header and footer
\pagestyle{fancy}
\rhead{\hmwkAuthorName} % Top left header
\lhead{\hmwkClass: \hmwkTitle} % Top center header
\rhead{\firstxmark} % Top right header
\lfoot{Łukasz Nowosielski} % Bottom left footer
\cfoot{} % Bottom center footer
\rfoot{} % Bottom right footer
\renewcommand{\headrulewidth}{1pt}
\renewcommand{\footrulewidth}{1pt}

    
\newcommand{\hmwkTitle}{Facebook Clone} % Tytuł projektu
\newcommand{\hmwkDueDate}{\today} % Data 
\newcommand{\hmwkClass}{Technologie chmurowe} % Nazwa przedmiotu
\newcommand{\hmwkAuthorName}{Łukasz Nowosielski} % Imię i nazwisko

% trabalho 
\begin{document}
% capa
\begin{titlepage}
    \vfill
	\begin{center}
	\hspace*{-1cm}
	\vspace*{0.5cm}
    \includegraphics[scale=0.55]{imagens/loga.png}\\
	\textbf{Uniwersytet Gdański \\ [0.05cm]Wydział Matematyki, Fizyki i Informatyki \\ [0.05cm] Instytut Informatyki}

	\vspace{0.6cm}
	\vspace{4cm}
	{\huge \textbf{\hmwkTitle}}\vspace{8mm}
	
	{\large \textbf{\hmwkAuthorName}}\\[3cm]
	
		\hspace{.45\textwidth} %posiciona a minipage
	   \begin{minipage}{.5\textwidth}
	   Projekt z przedmiotu technologie chmurowe na kierunku informatyka profil praktyczny na Uniwersytecie Gdańskim.\\[0.1cm]
	  \end{minipage}
	  \vfill
	%\vspace{2cm}
	
	\textbf{Gdańsk}
	
	\textbf{\hmwkDueDate}
	\end{center}
	
\end{titlepage}

\newpage
\setcounter{secnumdepth}{5}
\tableofcontents
\newpage

\section{Opis projektu}
\label{sec:Project}

Projekt "facebook-clone" powstał z myślą o dynamicznie rozwijającej się firmie \textbf{ConnectSphere}, która specjalizuje się w dostarczaniu innowacyjnych rozwiązań technologicznych. \textbf{ConnectSphere} zauważyła, że wiele małych i średnich przedsiębiorstw potrzebuje prostych, ale skutecznych narzędzi do budowania społeczności i komunikacji z klientami. Jednak popularne platformy społecznościowe, takie jak Facebook, często są zbyt obciążone nadmiarem funkcji, co prowadzi do wolnej i nieefektywnej komunikacji.

\textbf{ConnectSphere} potrzebowała prostego, ale skutecznego narzędzia do budowania społeczności wokół swojej misji. Chcieli stworzyć miejsce, gdzie firmy mogłyby łatwo komunikować się ze swoimi klientami, dzielić się aktualnościami i informacjami o produktach, a także otrzymywać bezpośrednie opinie.

Dlatego powstał pomysł na stworzenie "facebook-clone" - platformy, która oferuje podstawowe funkcje społecznościowe, takie jak tworzenie profilu, udostępnianie postów, komentowanie i reagowanie na posty innych użytkowników. Projekt ma na celu dostarczenie tych funkcji w prosty i intuicyjny sposób, umożliwiając \textbf{ConnectSphere} skuteczną komunikację i budowanie społeczności wokół ich produktów i usług.

\subsection{Opis architektury - 8 pkt}
\label{sec:introduction}
"facebook-clone" jest zbudowane z interfejsu użytkownika (frontend), serwera aplikacji (backend), bazy danych oraz serwisu uwierzytelniania Keycloak.

Aplikacja jest hostowana na klastrze Kubernetes, który zapewnia skalowalność, niezawodność i wydajność. Interfejs użytkownika jest dostępny dla użytkowników za pośrednictwem przeglądarki internetowej, podczas gdy serwer aplikacji komunikuje się z bazą danych mongodb i obsługuje żądania użytkowników. Elementy te są zabezpieczone dzięki integracji z Keycloak.

Aplikacja korzysta z Ingress w Kubernetes, który zarządza dostępem do usług w klastrze, zapewniając wydajne i niezawodne przekierowanie ruchu. Dzięki Ingress, aplikacja jest dostępna pod adresem https://facebook-clone.com (frontend), https://api.facebook-clone.com (backend) i https://keycloak.facebook-clone.com (keycloak). Ingress umożliwia także łatwe zarządzanie ruchem sieciowym, co jest kluczowe dla utrzymania wydajności i niezawodności aplikacji.

\subsection{Opis infrastruktury - 6 pkt}
\label{sec:Users}

Aplikacja "facebook-clone" jest zaprojektowana do działania w środowisku Kubernetes, wykorzystując Minikube. Minikube to narzędzie, które umożliwia uruchomienie jednego węzła klastra Kubernetes na lokalnym komputerze. Jest to idealne rozwiązanie dla deweloperów, którzy chcą testować swoje aplikacje w środowisku Kubernetes bez konieczności tworzenia pełnego klastra. Specyfikacja klastra Minikube to 4GB pamięci RAM, 2 vCPU i 10GB miejsca na dysku.

\subsection{Opis komponentów aplikacji - 8 pkt}
\label{sec:FunctionalConditions}

Aplikacja składa się z:

\begin{itemize}
\item \textbf{Frontend:} Interfejs użytkownika został zbudowany za pomocą biblioteki React. Do serwowania statycznych plików używamy serwera Nginx. Kod źródłowy frontendu jest zawarty w kontenerze Docker, który jest konfigurowany i uruchamiany za pomocą Dockerfile.

\item \textbf{Backend:} Backend został zbudowany za pomocą frameworka Express.js. Backend komunikuje się z bazą danych MongoDB, która jest hostowana na platformie MongoDB Atlas. Podobnie jak frontend, backend jest również zawarty w kontenerze Docker, który jest konfigurowany i uruchamiany za pomocą Dockerfile.

\item \textbf{Baza danych:} Aplikacja korzysta z MongoDB jako bazy danych. Jest ona hostowana w chmurze, co zapewnia dodatkową niezawodność i elastyczność. Baza danych jest skonfigurowana do komunikacji z serwerem aplikacji za pośrednictwem bezpiecznego połączenia.

\item \textbf{Keycloak:} Do zarządzania uwierzytelnianiem i autoryzacją w aplikacji używamy Keycloak. Keycloak jest skonfigurowany do pracy w trybie "client credentials", gdzie autoryzacja jest oparta na identyfikatorze i tajnym kluczu klienta, a nie na danych użytkownika. Tokeny uwierzytelniające są generowane przez Keycloak i przekazywane do backendu, gdzie są używane do zabezpieczania naszego API.

\item \textbf{PostgreSQL:} Aplikacja korzysta z bazy danych PostgreSQL, która jest zarządzana przez Keycloak. Baza danych jest przechowywana na stałym woluminie (PVC), co pozwala na zachowanie danych oraz ustawień dla Keycloak nawet po zatrzymaniu i ponownym uruchomieniu kontenera.
\end{itemize}

\subsection{Konfiguracja i zarządzanie - 4 pkt}
\label{sec:NonFunctionalConditions}

Konfiguracja i zarządzanie aplikacją na poziomie klastra Kubernetes odbywa się na kilka sposobów:

\begin{itemize}
\item \textbf{Deploymenty:} Każdy komponent aplikacji (frontend, backend, Keycloak, baza danych PostgreSQL) jest uruchamiany jako osobny Deployment w Kubernetes. Deploymenty pozwalają na łatwe skalowanie i aktualizację komponentów aplikacji.

\item \textbf{Serwisy:} Komunikacja między komponentami aplikacji jest zarządzana za pomocą Serwisów Kubernetes, które zapewniają odkrywanie i balansowanie obciążenia.

\item \textbf{ConfigMaps i Secrets:} Konfiguracja aplikacji jest przechowywana w ConfigMaps (Ingress, Keycloak, PostgreSQL) i Secrets (Backend, PostgreSQL). Pozwala to na centralne zarządzanie konfiguracją i bezpieczne przechowywanie poufnych danych.

\item \textbf{Persistent Volume Claims:} Dane bazy danych PostgreSQL są przechowywane na Persistent Volume za pomocą Persistent Volume Claim, co zapewnia trwałość danych.

\item \textbf{Ingress:} Dostęp do aplikacji z zewnątrz klastra jest zarządzany za pomocą Ingress, który kieruje ruch sieciowy do odpowiednich Serwisów.
\end{itemize}

\subsection{Zarządzanie błędami}
\label{sec:ERD} 

\begin{itemize}
\item \textbf{Obsługa błędów na poziomie aplikacji:} W backendzie i frontendzie korzystam z bloków `try {} catch {}` do przechwytywania i obsługi błędów. Dzięki temu, nawet w przypadku nieoczekiwanych błędów, aplikacja nie przestaje działać.
\item \textbf{Zarządzanie błędami na poziomie infrastruktury:} W konfiguracji Deployment dla backendu, korzystam z mechanizmu replikacji Kubernetes. Dzięki temu, jeśli jeden z podów aplikacji napotka błąd i przestanie działać, Kubernetes automatycznie przekieruje ruch do drugiego poda.

\end{itemize}

\subsection{Skalowalność - 4 pkt}
\label{sec:ExamplesSection}

Skalowalność jest kluczowa w architekturze aplikacji opartej na Kubernetes.

\begin{itemize}
\item \textbf{Skalowanie horyzontalne:} Aplikacja jest skalowana horyzontalnie poprzez zwiększanie liczby replik podów na backendzie. Obecnie liczba replik na backendzie została zwiększona do 2, co pozwala na lepsze rozłożenie obciążenia i zwiększa przepustowość. Kubernetes automatycznie rozprowadza ruch między replikami, zapewniając równomierne obciążenie.

\end{itemize}

\subsection{Wymagania dotyczące zasobów - 2 pkt}
\label{sec:ExampleTables}

\begin{itemize}
\item \textbf{Frontend:} 300MB pamięci RAM, 0.2 rdzenia CPU
\item \textbf{Backend:} 600MB pamięci RAM, 0.2 rdzenia CPU (na replikę)
\item \textbf{PostgreSQL:} 256MB pamięci RAM, 0.15 rdzenia CPU
\item \textbf{Keycloak:} 2000MB pamięci RAM, 1 rdzeń CPU
\end{itemize}


\subsection{Architektura sieciowa - 4 pkt}
\label{sec:ExampleResults}

Architektura sieciowa aplikacji opiera się na modelu sieciowym Kubernetes, który umożliwia komunikację między podami oraz z zewnętrznymi usługami sieciowymi.

\begin{itemize}
\item \textbf{Ingress:} Wszystkie usługi (frontend, backend, Keycloak) są dostępne poprzez Ingress, który zarządza dostępem do usług w klastrze z zewnątrz. Ingress umożliwia routowanie ruchu sieciowego do odpowiednich usług na podstawie adresu URL i metody HTTP.

\item \textbf{Protokoły:} Aplikacja korzysta z protokołu HTTP do komunikacji między frontendem a backendem. Do zarządzania stanem sesji i autoryzacji używany jest protokół OAuth2 z Keycloak.

\end{itemize}

\nocite{*}

\end{document}